% Created 2020-10-01 Kam 13:29
% Intended LaTeX compiler: pdflatex
\documentclass[11pt]{article}
\usepackage[utf8]{inputenc}
\usepackage[T1]{fontenc}
\usepackage{graphicx}
\usepackage{grffile}
\usepackage{longtable}
\usepackage{wrapfig}
\usepackage{rotating}
\usepackage[normalem]{ulem}
\usepackage{amsmath}
\usepackage{textcomp}
\usepackage{amssymb}
\usepackage{capt-of}
\usepackage{hyperref}
\author{Aziz Faozi}
\date{\today}
\title{GFW}
\hypersetup{
 pdfauthor={Aziz Faozi},
 pdftitle={GFW},
 pdfkeywords={},
 pdfsubject={},
 pdfcreator={Emacs 26.3 (Org mode 9.1.9)}, 
 pdflang={English}}
\begin{document}

\maketitle
\tableofcontents



\section{Pengantar}
\label{sec:orgdb9e30b}
\section{Desain}
\label{sec:org3656973}
\subsection{Format Pesan}
\label{sec:org8dc29b2}
\subsubsection{v1}
\label{sec:org6b0d1f1}
untuk API versi pertama mengunakan perintah berikut ini
\begin{verbatim}

{
"id":"5",
"device":"dev112",
"longitude":123.321,
"latitude":321.12,
"dateStamp":"datetimebentukstring"
}

\end{verbatim}
Dengan command seperti berikut 
\begin{verbatim}
curl -X POST http://localhost:8080/api-gfw/v1/add \
-H 'Content-type:application/json' -d \
'{"id":"100", "device": "dev112", "longitude": \
 123.321, "latitude": 321.12, \
"DateStamp": "ngaceng"}'
\end{verbatim}

\subsubsection{v2}
\label{sec:orga349de1}
Untuk perbaikan di versi pertama dengan tetap mempertahankan 
versi pertama maka format pesan di buat pada v2 adalah sebagai berikut
\begin{verbatim}
{
"device":"device1234",
"data": {
"position": {
"longitude": [123.321, 123.321, ...]
"latitude": [321.321, 123.321, ...]
"datestamp": ["pos1", "pos2", ...]
}
}
}
\end{verbatim}
\section{Logbook}
\label{sec:orgfed0cc3}
\subsection{Masalah}
\label{sec:orge3f5cea}

\subsubsection{\textit{<2020-10-01 Kam 12:31>}}
\label{sec:org052c2e2}
Masih ada masalah dengan methode insert id di api. 
\end{document}
