% Created 2020-11-23 Mon 19:37
% Intended LaTeX compiler: pdflatex
\documentclass[11pt]{article}
\usepackage[utf8]{inputenc}
\usepackage[T1]{fontenc}
\usepackage{graphicx}
\usepackage{grffile}
\usepackage{longtable}
\usepackage{wrapfig}
\usepackage{rotating}
\usepackage[normalem]{ulem}
\usepackage{amsmath}
\usepackage{textcomp}
\usepackage{amssymb}
\usepackage{capt-of}
\usepackage{hyperref}
\date{\today}
\title{GFW}
\hypersetup{
 pdfauthor={},
 pdftitle={GFW},
 pdfkeywords={},
 pdfsubject={},
 pdfcreator={Emacs 26.3 (Org mode 9.1.9)}, 
 pdflang={English}}
\begin{document}

\maketitle
\tableofcontents



\section{Pengantar}
\label{sec:orgb905b9a}
\section{Desain}
\label{sec:org0922b7e}

\subsection{Format Pesan}
\label{sec:orgb47b16b}
\subsubsection{Contoh data}
\label{sec:orgad2ce8a}
Contoh data yang di tampilkan dalam bentuk JSON
\begin{verbatim}
{ 
"id":"27d48518-218a-11eb-9790-3f50feae72b1",
"online":1.6048162822154794E9,
"status":1.0,
"sateliteUsed":15.0,
"mode":3.0,
"time":1.604816282E9,
"altitude":199.31,
"speed":0.0,
"track":96.5,
"pdop":1.3,
"device":"DRAGON001",
"longitude":106.725318833,
"latitude":-6.555990667,
"dateStamp":"Sun Nov  8 06:18:02 2020\n"
}
\end{verbatim}
dari data tersebut anda bisa melihat beragam variable berikut penjelasannya
\begin{center}
\begin{tabular}{ll}
\hline
variable & keterangan\\
\hline
id & ID merupakan Identitas dari data yang dikirim. Nilai ini dibuat\\
 & dengan menggunakan UUID.\\
online & Merupakan variable yang dihasilkan dari status online pada gpsmon\\
 & nilai ini juga memberikan nilai waktu.\\
status & Status gps\\
sateliteUsed & Merupakan jumlah satelite yang digunakan.\\
mode & \\
time & Merupakan waktu yang digunakan oleh GPS untuk menerima data.\\
 & Waktu ini dalam format float.\\
altitude & Nilai ketinggian dari data GPS,\\
speed & Kecepatan perangkat yang tertangap GPS.\\
track & \\
pdop & \\
longitude & Nilai longitude yang diberikan GPS.\\
latitude & nilai latitude yang diberikan GPS.\\
dateStamp & Merupakan waktu kirim data dari device ke server.\\
\hline
\end{tabular}
\end{center}

\subsection{Link API pada server}
\label{sec:org8db0090}
\subsubsection{Link API untuk GET semua data}
\label{sec:org029f415}
Untuk mendapatkan data dari seluruh data yang tersimpan pada
server anda bisa melakukan transaksi seperti berikut.
\begin{center}
\begin{tabular}{ll}
\hline
URL & \url{http://server.faozi.tech:8083/api-gfw/v1/all}\\
METHODE & GET\\
RESPONSE & list Semua Data\\
\end{tabular}
\end{center}

\subsubsection{Link API untuk POST data}
\label{sec:orge262a3d}
Untuk menambahkan data GPS pada server anda bisa melakukan
transaksi dengan metode seperti berikut
\begin{center}
\begin{tabular}{ll}
URL & \url{http://server.faozi.tech:8083/api-gfw/v1/add}\\
METHODE & POST\\
REQ\(_{\text{BODY}}\) & \{\\
 & "id":"27d48518-218a-11eb-9790-3f50feae72b1",\\
 & "online":1.6048162822154794E9,\\
 & "status":1.0,\\
 & "sateliteUsed":15.0,\\
 & "mode":3.0,\\
 & "time":1.604816282E9,\\
 & "altitude":199.31,\\
 & "speed":0.0,\\
 & "track":96.5,\\
 & "pdop":1.3,\\
 & "device":"DRAGON001",\\
 & "longitude":106.725318833,\\
 & "latitude":-6.555990667,\\
 & "dateStamp":"Sun Nov  8 06:18:02 2020\n"\\
 & \}\\
RESPONSE & Sama seperti REQ\(_{\text{BODY}}\)\\
\end{tabular}
\end{center}
\subsubsection{Link untuk Status Server}
\label{sec:org021b112}
Untuk keperluan tertentu anda bisa melakukan checking server
dengan melakukan GET pada URL berikut
\begin{center}
\begin{tabular}{ll}
URL & \url{http://server.faozi.tech:8083/api-gfw/v1/status}\\
METHODE & GET\\
RESPONSE & \{"kode":1,"status":"OK"\}\\
\end{tabular}
\end{center}

\section{Logbook}
\label{sec:org37b6378}
\subsection{Masalah}
\label{sec:org5ae01e7}
\subsubsection{AutoInsert API}
\label{sec:org580f8c0}
Tanggal kasus : \textit{<2020-10-01 Thu 12:31> } . 
Deskripsi : Masih ada masalah dengan methode insert id di api. 
Setiap insert data belum bisa autoincremenet id. 
\end{document}
