% Created 2020-11-23 Mon 21:41
% Intended LaTeX compiler: pdflatex
\documentclass[11pt]{article}
\usepackage[utf8]{inputenc}
\usepackage[T1]{fontenc}
\usepackage{graphicx}
\usepackage{grffile}
\usepackage{longtable}
\usepackage{wrapfig}
\usepackage{rotating}
\usepackage[normalem]{ulem}
\usepackage{amsmath}
\usepackage{textcomp}
\usepackage{amssymb}
\usepackage{capt-of}
\usepackage{hyperref}
\author{Aziz Amerul Faozi}
\date{\today}
\title{GFW Doc}
\hypersetup{
 pdfauthor={Aziz Amerul Faozi},
 pdftitle={GFW Doc},
 pdfkeywords={},
 pdfsubject={},
 pdfcreator={Emacs 26.3 (Org mode 9.1.9)}, 
 pdflang={English}}
\begin{document}

\maketitle
\tableofcontents


\section{Pengantar}
\label{sec:orgcb7f619}
Dokument ini merupakan penjelasan kerja dan beragam setting configurasi serta 
lokasi log error untuk melakukan perbaikan atau ketika menjalankan aplikasi
yang ditanamkan pada dragonboard ataupun aplikasi yang dijalakan di server.
\section{Desain}
\label{sec:orgfdb7b7a}
\subsection{Client}
\label{sec:org935fa69}
Untuk Clientnya dibangun dengan menggunakan bahasa C. Menggunakan pthread, curl, sqlite3,
dan libgpsd
\subsubsection{Run Time}
\label{sec:orgbc08caa}
Karena menggunakan dinamic linking maka beberapa runtime harus tersedia di sbc dragonboard.
Tetapi jika anda ingin melakukan development langsung pada device maka berikut  yang harus 
diinstall.

\begin{enumerate}
\item GNU build system (autoconf, automake, libtool)
\item Curl
\item pthread
\item sqlite3
\end{enumerate}
\subsubsection{Install dependency}
\label{sec:org8404cb3}
\begin{verbatim}
sudo apt-get install cmake make unzip
sudo apt-get install libcurl4-openssl-dev
sudo apt-get install libgps-dev libsqlite3-dev
sudo apt-get install qdsp-start
wget  https://s3.amazonaws.com/json-c_releases/releases/json-c-0.15.tar.gz
tar -zxvf json-c-0.15.tar.gz
cd json-c-0.15
mkdir build &&
cd    build &&

cmake -DCMAKE_INSTALL_PREFIX=/usr \
      -DCMAKE_BUILD_TYPE=Release \
      -DBUILD_STATIC_LIBS=OFF    \
      .. &&
make && sudo make install
\end{verbatim}
\subsubsection{Clone repository}
\label{sec:org332e228}
\begin{verbatim}
wget https://github.com/faoziaziz/api-gfw/archive/latest.zip
unzip latest.zip
cd api-gfw-latest/
\end{verbatim}
\subsubsection{Membuat directory installer}
\label{sec:org849e891}
sebelum membuat membuild program pertama buatlah directory
\begin{verbatim}
sudo mkdir -p /opt/gfw 
sudo chown $USER:$USER /opt/gfw/
\end{verbatim}
\subsubsection{Untuk membuild program}
\label{sec:orgb8f2311}
\begin{verbatim}
autoreconf -i
./configure --prefix=/opt/gfw/
make 
make install
\end{verbatim}
setelah terinstall maka akan ada file 
\begin{verbatim}
/opt/gfw/bin/api-gfw-client
\end{verbatim}
dengan mengeksekusi file tersebut maka program akan berjalan.
\subsubsection{Log Error}
\label{sec:org260428d}
log error tersedia pada 
\begin{verbatim}
/opt/gfw/error.log
\end{verbatim}
\subsubsection{Database Device}
\label{sec:orgd6d7b0b}
Jika tidak terdapat koneksi untuk mengirimkan data ke server maka file akan disimpan pada 
database local 
\begin{verbatim}
/opt/gfw/gfw_data_gps.db
\end{verbatim}
\subsubsection{Konfigurasi Lainnya}
\label{sec:org7ab5aa8}
Anda juga bisa melakukan setting terhadap nama device, link url API atau  lokasi database local
dengan mengedit file src/conf.h pada code yang di download.
\begin{verbatim}
#ifndef __conf_h
#define __conf_h

/* this debug setting */
#define DEBUG 1
/* url api configuration*/
#define URL_API "http://server.faozi.tech:8083/api-gfw/v1"

#define DEVICE "DRAGON001"
#define DATABASE_LOC "/opt/gfw/gfw_data_gps.db"

#endif
\end{verbatim}
\subsection{Server}
\label{sec:orgea730b5}
Server menggunakan Spring-boot untuk aplikasi pengolah data dan Redis sebagai data
storagenya. Data yang dikirimkan dari device menuju server menggunakan perantara 
JSON. 
\subsubsection{Contoh data}
\label{sec:orga5539fc}
Contoh data yang di tampilkan dalam bentuk JSON
\begin{verbatim}
{ 
"id":"27d48518-218a-11eb-9790-3f50feae72b1",
"online":1.6048162822154794E9,
"status":1.0,
"sateliteUsed":15.0,
"mode":3.0,
"time":1.604816282E9,
"altitude":199.31,
"speed":0.0,
"track":96.5,
"pdop":1.3,
"device":"DRAGON001",
"longitude":106.725318833,
"latitude":-6.555990667,
"dateStamp":"Sun Nov  8 06:18:02 2020\n"
}
\end{verbatim}
dari data tersebut anda bisa melihat beragam variable berikut penjelasannya
\begin{center}
\begin{tabular}{ll}
\hline
variable & keterangan\\
\hline
id & ID merupakan Identitas dari data yang dikirim. Nilai ini dibuat\\
 & dengan menggunakan UUID.\\
online & Merupakan variable yang dihasilkan dari status online pada gpsmon\\
 & nilai ini juga memberikan nilai waktu.\\
status & Status gps\\
sateliteUsed & Merupakan jumlah satelite yang digunakan.\\
mode & \\
time & Merupakan waktu yang digunakan oleh GPS untuk menerima data.\\
 & Waktu ini dalam format float.\\
altitude & Nilai ketinggian dari data GPS,\\
speed & Kecepatan perangkat yang tertangap GPS.\\
track & \\
pdop & \\
longitude & Nilai longitude yang diberikan GPS.\\
latitude & nilai latitude yang diberikan GPS.\\
dateStamp & Merupakan waktu kirim data dari device ke server.\\
\hline
\end{tabular}
\end{center}

\subsection{Link API pada server}
\label{sec:org6854931}
\subsubsection{Link API untuk GET semua data}
\label{sec:org380f757}
Untuk mendapatkan data dari seluruh data yang tersimpan pada
server anda bisa melakukan transaksi seperti berikut.
\begin{center}
\begin{tabular}{ll}
\hline
URL & \url{http://server.faozi.tech:8083/api-gfw/v1/all}\\
METHODE & GET\\
RESPONSE & list Semua Data\\
\end{tabular}
\end{center}

\subsubsection{Link API untuk POST data}
\label{sec:org8cd5396}
Untuk menambahkan data GPS pada server anda bisa melakukan
transaksi dengan metode seperti berikut
\begin{center}
\begin{tabular}{ll}
URL & \url{http://server.faozi.tech:8083/api-gfw/v1/add}\\
METHODE & POST\\
REQ\(_{\text{BODY}}\) & \{\\
 & "id":"27d48518-218a-11eb-9790-3f50feae72b1",\\
 & "online":1.6048162822154794E9,\\
 & "status":1.0,\\
 & "sateliteUsed":15.0,\\
 & "mode":3.0,\\
 & "time":1.604816282E9,\\
 & "altitude":199.31,\\
 & "speed":0.0,\\
 & "track":96.5,\\
 & "pdop":1.3,\\
 & "device":"DRAGON001",\\
 & "longitude":106.725318833,\\
 & "latitude":-6.555990667,\\
 & "dateStamp":"Sun Nov  8 06:18:02 2020\n"\\
 & \}\\
RESPONSE & Sama seperti REQ\(_{\text{BODY}}\)\\
\end{tabular}
\end{center}
\subsubsection{Link untuk Status Server}
\label{sec:org1449cdf}
Untuk keperluan tertentu anda bisa melakukan checking server
dengan melakukan GET pada URL berikut
\begin{center}
\begin{tabular}{ll}
URL & \url{http://server.faozi.tech:8083/api-gfw/v1/status}\\
METHODE & GET\\
RESPONSE & \{"kode":1,"status":"OK"\}\\
\end{tabular}
\end{center}

\section{Logbook}
\label{sec:org76cbdf3}
\subsection{Masalah}
\label{sec:org9f011b6}
\subsubsection{AutoInsert API}
\label{sec:org7146d5c}
Tanggal kasus : \textit{<2020-10-01 Thu 12:31> } . 
Deskripsi : Masih ada masalah dengan methode insert id di api. 
Setiap insert data belum bisa autoincremenet id. 
\end{document}
